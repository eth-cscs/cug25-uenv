This section presents the components of CSCS's software deployment workflow, following the complete lifecycle from initial software compilation through to end-user interaction.
We examine three key stages: the building of software components with their specific dependencies and optimizations, the deployment mechanisms across diverse computational environments, and the CLI and SLURM tools that provide these environments to users.
%This structured approach allows us to thoroughly address the complexities inherent in managing scientific software on large-scale HPC systems.

%------------------------------------------------------------------------------
\subsection{Building Software: Stackinator}
%------------------------------------------------------------------------------

The  \stackinator tool, that is used to generate squashfs images from a YAML-based recipe, was documented in detail at CUG 2023~\cite{uenv2023}.
The implementation details of \stackinator were covered in that paper, and have not changed materially.
The method used by \stackinator to install \craympich using Spack~\cite{gamblin:sc15} outside the CPE was demonstrated was shown in~\cite{uenv2023} 
This approach is used by \stackinator, however it can also be used directly with Spack -- CSCS staff use it to install the most recent versions of \craympich on LUMI.

\todo{add some more color about stackinator: changes since 2023; per-stack spack versioning; site repo; how all this minimises the Spack upgrade pain.}

\todo{providing libfabric, cxi, cassini}

\todo{OpenMPI}

\todo{NVIDIA libraries}

This section will focus on Spack packing for other key libraries that implement inter-node communication -- for example MPI, NCCL and \nvshmem -- need to be optimized for the Slingshot 11 network in HPE Cray-EX systems.

We will focus on three main sets of libraries. The first is building OpenMPI with libfabric support, in order to provide an alternative to \craympich when there are bugs or performance regressions for specific applications, and to support software that is distributed as binaries linked against OpenMPI. The second is how to build the open source libfabric and CXI driver software for Slingshot 11. Finally, we will show how we adapt \cufftmp and \cusolvermp, which are distributed as pre-build binaries for infiniband, to use \nvshmem with libfabric support, which we developed in collaboration with NVIDIA.

Note that while these methods are integrated into Stackinator, they can be used directly by Spack. All Spack packages, scripts and guides will be made available on GitHub for readers to use.

%------------------------------------------------------------------------------
\subsection{Deployment: CI/CD}
\label{sec:cicd}
%------------------------------------------------------------------------------
Alps has five node node types described in \tbl{tab:alps-nodes}, on top of which CSCS creates vClusters\footnote{Versatile software defined cluster}~\cite{vClusters2023}, where each vCluster is a separate SLURM cluster that is customised for a tenant.
As such, the version of SLURM, mounted file systems and software installed in the OS image (including libfabric) can vary between vClusters.

\begin{table*}[!htb]
    \begin{minipage}{0.6\textwidth}
        \centering
        \begin{tabular}{llrrrr}
        \toprule
        uarch   & type         & blades & nodes & CPU sockets & GPU devices \\
        \midrule
        gh200   & NVIDIA GH200 & 1,344   & 2,688  & 10,752      & 10,752      \\
        zen2    & AMD Rome     & 256     & 1,024  & 2,048       & --          \\
        a100    & NVIDIA A100  & 72      & 144    & 144         & 576         \\
        mi300   & AMD MI300A   & 64      & 128    & 512         & 512         \\
        mi200   & AMD MI250x   & 12      & 24     & 24          & 96          \\
        \midrule
        \multicolumn{2}{c}{\textsc{Total}}      & 1,748   & 3,880  & 13,480  & 11,936 \\
        \bottomrule
        \end{tabular}
    \end{minipage}%
    \begin{minipage}{0.4\textwidth}
        \centering
        \begin{tabular}{lll}
        \toprule
        tenant   & vCluster & uarch         \\
        \midrule
            ML      & Clariden & gh200 \\
            ML      & Bristen  & a100 \\
            HPC     & Daint    & gh200 \\
            HPC     & Eiger    & zen2 \\
            Climate & Santis   & gh200  \\
            MetoSwiss & Balfrin   & a100 and zen2  \\
        \bottomrule
        \end{tabular}
    \end{minipage}
    \caption{Alps node types and their specifications (left), and examples of vClusters provided to tenants (right).}
\label{tab:alps-nodes}
\end{table*}

Over the last two years uenv have proven to be portable between vClusters with the same microarchitecture, and we continue to further reduce the dependence on the underlying OS image, which will also help improve long-term stability of software stacks.
However, to deploy a uenv to a vCluster we still require that it be built on a compute node of the target vCluster to avoid the need for cross compilation, and to ensure that dependencies like libfabric and Slurm are correctly handled.

Manually building and deploying uenv images would be tedious and error prone.
We use a CI/CD pipeline 
\begin{itemize}
    \item The uenv image recipes (YAML files) are maintained in a GitHub repository.
    \item Comments on pull requests trigger a pipeline that builds and tests the uenv image: e.g \lst{system=daint;uarch=gh200;uenv=vasp:v6.5.0}
    \item A GitLab runner that launches build and test stages on the target cluster.
    \item A ReFrame test suite that selects the appropriate tests to run after building the uenv.
\end{itemize}

\todo{expand the points below. Provide an example of the oras commands used to push, pull and attach meta data}

The build pipeline generates two artifacts: a squashfs image and image meta data, which are pushed to an on-premises container registry.
The registry is provided by JFrog\footnote{\href{https://jfrog.com}{\lst{jfrog.com}}}, and Oras\footnote{\href{https://oras.land}{\lst{oras.land}}} is used to push and pull images, so any DockerHub API compatible registry would work.
The images are stored in a \lst{build} namespace, with a tag that corresponds to the unique id of the CI/CD job that built the image, for example:
\begin{lstlisting}
jfrog.svc.cscs.ch/uenv/build/daint/gh200/vasp/v6.5.0:1631426005
\end{lstlisting}

The final deployment is manual, because we find that it is often necessary to first perform additional testing such as providing the image to selected users for validation.
Deployment is performed using the CLI tool (see \sect{sec:cli}):
\begin{lstlisting}
uenv image copy build::vasp:1631426005 \
                deploy::vasp/v6.5.0:v1@daint
\end{lstlisting}
Once deployed, the image is available for users to pull and run.

Users can also use the same build pipeline to build their own uenv from a recipe from anywhere on the CSCS network:
\begin{lstlisting}
> uenv build myapp/v1.2@daint%gh200 <recipe-path>
Log         : https://cicd-ext-mw.cscs.ch/ci/uenv/
build?image=cu3upvpoag1s73eo3n80-3690753405420143
Status      : submitted

Destination
Registry    : jfrog.svc.cscs.ch/uenv
Namespace   : service
Label       : myapp/v1.2:1626811672@daint%gh200
\end{lstlisting}
where \lst{recipe-path} is the YAML recipe.
The link can be used to check the status of the image, and once the image has been built the user can pull it:
\begin{lstlisting}
uenv image pull service::myapp/v1.2:1626811672
\end{lstlisting}




%------------------------------------------------------------------------------
\subsection{User Experience: CLI and SLURM}
\label{sec:cli}
%------------------------------------------------------------------------------
A command line tool called \emph{uenv} is used to interact with the SquashFS images.

CSCS provides a command line tool for users to manage uenv images, and start sessions.

The uenv tool is written in C++, with a library that is also shared with the SLURM plugin discussed in \sect{sec:slurm}.
Source code is available in an open source GitHub repository\footnote{\href{https://github.com/eth-cscs/uenv2}{\lst{github.com/eth-cscs/uenv2}}}.


Static binary and .so file, with CI/CD that build RPMs ready to install on HPE systems.

%---------------------------------------------------
\miniheader{Squashfs-Mount}

Non-privileged users are able to mount SquashFS images at runtime using the \lst{squashfs-mount} command line utility, which is a small \lst{setuid} executable that creates a new mount namespace, mounts the SquashFS file through \lst{libmount}, drops privileges and executes a given command.
This procedure is very similar to SquashFS-based HPC container runtimes such as Apptainer and Sarus.

The following example starts a bash shell with the Squashfs files \lst{img1.sqfs} and \lst{img2.sqfs} are mounted at \lst{/mnt1} and \lst{/mnt2} respectively:
\lstinputlisting[language=bash]{code/squashfs-mount.sh}

The utility is open source, \href{https://github.com/eth-cscs/squashfs-mount}{available on GitHub} and includes RPMs for installation on Cray EX.

%---------------------------------------------------
\miniheader{Image Management}

uenv images are stored in a \emph{repository} -- a directory with an sqlite database at the root, and uenv images stored in an images sub-directory:

%\begin{figure}[htp!]
\dirtree{%
.1 \$SCRATCH/.uenv-images.
.2 index.db.
.2 images.
.3 01edd4\dots76a5ab.
.4 store.squashfs.
.4 env.json.
.4 config.json.
.4 recipe.
.5 \dots.
.4 extra.
.5 reframe.yaml.
.3 e7b0d9\dots47c377.
.4 \dots.
.3 e7e508\dots254742.
.4 \dots.
}


%\caption{caption text.}
%\label{fig:repo-path}
%\end{figure}

Each image is stored in a path that matches its hash, with the following information:
\begin{itemize}
\item  \lstinline{store.squashfs}: the SquashFS file;
\item  \lstinline{env.json}: information about the mount point, uenv description, and view discriptions (their name, and the environment variable updates that they aplly).
\item  \lstinline{recipe}: the original recipe used to build the image;
\item  \lstinline{extra/reframe.yaml}: description of the testable features the uenv implements, and how to configure the environment to test them.
\end{itemize}

The uenv command line tool (uenv CLI) allows users to interact with and manage uenv images.
Below is an example workflow where a user first discovers which images are available, downloads an image, then runs a shell with the image running.

\lstinputlisting[language=bash]{code/uenv-examples.sh}

The uenv provide interfaces called views, that set environment variables.
The \emph{modules} view will load modules, the \emph{spack} view simplifies using Spack to build software on top of the packages provided in the uenv, and uenv authors can also provide custom views that make subsets of the software in the uenv available.

%---------------------------------------------------
\miniheader{Deployment}

%---------------------------------------------------
\miniheader{Lessons Learnt}

The first version of uenv was written in Python, and was able to modify the calling environment similarly to modules.
For example, the following command would set environment variables for a uenv that was running:
\lstinputlisting[language=bash]{code/uenv-old-view.sh}

A full rewrite of uenv was conducted, based on the lessons learnt from this first version.

The first lesson, which is somewhat subjective, is that Python was not an ideal language for deploying to production for the following reasons:
\begin{itemize}
    \item the call to Python was designed to use Python 3.6 installed as part of SUSE, and had to be isolated;
    \item the project was deployed as a set of directories containing the Python implementating
    \item language features like the lack of type safety and exception-based error handling made implementing a robust, error-free code difficult as the size of the implementation increased.
\end{itemize}

This was simplified greatly by installing a single statically-linked C++ executable, written in C++20 to take advantage of modern language features for error handling, filesystem operations, and sanitizers.

The second lesson was that being able to modify the calling environment over-complicated the implementation, and enabled some bad practices.

In order to modify the calling environment, the \lstinline{uenv} command was bash function that forwarded the arguments to the Python implementation: \lstinline{echo "$($UENV_CMD $flags "$@")"}, where \lstinline{UENV_CMD} is the path of the implementation.
The implementation would process the arguments, and print a series of shell commands to stdout, which the calling bash function would then exec.

For example, the command \lstinline{uenv view default} would lead to the following commands being echoed:
\lstinputlisting[language=bash]{code/uenv-old-echo.sh}

The wrapper-based implementation made it difficult to debug, required further wrappers inside calls to \lstinline{uenv start} and \lstinline{uenv run}, and made support for different shells difficult -- the original implementation only supported bash.
The new implementation only allows setting views when the uenv is loaded, which is performed by creating a new \lstinline{const char* environ} array that defines the new environment, and forwarding this to \lstinline{execvpe}.
This has the benefits:
\begin{itemize}
    \item the implementation is significantly simplified;
    \item it is shell agnostic - there is no bash-specific code anywhere in the implementation.
\end{itemize}

The final benefit is more subtle: specifying the target environment up front is declarative.
User tickets are easier to debug because the views they are loading are explicitly listed as flags to the uenv call, and all environment modifications are captured in the logs of the \lstinline{uenv start}, \lstinline{uenv run} or SLURM \lstinline{srun --uenv=? --view=?} calls.

