\todo{Describe "traditional" CPE deployment and why this is not compatible with the software deployment Objectives.}

The Cray Programming Environment (CPE) provides a rich software stack: compilers, communication libraries, commonly used scientific libraries (e.g. fftw, hdf5, netcdef) and tools.
The CPE provided by HPE as a collection of RPMs, that CSCS installed via Ansible and deployed to the nodes using the data virtualization service (DVS)\cite{dvs}.


\begin{itemize}
    \item it does not provide everything that all users require - sites typically install additional software based on CPE.
    \item Until recently CSCS has also provided software on top.
    \item release schedule is now every 6 months
    \item Downside: using CPE as a base layer breaks "no root", "no reboot", and "don't break user codes" tenents
    \item for these reasons CSCS no longer uses CPE as the base layer for software installation - we use methods in previous sections
    \item We still deployed an old version - hidden behind a cray module
    \item when users log in the first thing they do is module avail, and naturally choose the cray module.
    \item in this section we will describe how we use containers to deploy CPE for users who require it, in a way that does not violate our axioms.
\end{itemize}

Installing and updating CPE requires root permission to rebuilding node image and reboot nodes with the new image

and updating the installation of CPE can break user-installed software built on top of the CPE.
This is in opposition to the objective of giving support staff the ability to deploy software stacks without root or interruption to the system.


As of the time of writing, CPE is still available through a \lstinline{cray} module on Daint, though it is not "supported" on the system.

In \sect{sec:cpe-container} we describe how CSCS will continue providing CPE in a container.
