A command line tool called \emph{uenv} is the main interface for uenv users, alongside the SLURM plugin which is documented in \sect{sec:slurm}.

The uenv tool and SLURM plugin are written in C++, with a library that is used by both, in an open-source GitHub repository\footnote{\href{https://github.com/eth-cscs/uenv2}{\lst{github.com/eth-cscs/uenv2}}}.
The uenv command line tool is a statically-linked executable, and is bundled with the SLURM plugin in a single RPM, for ease of installation.

%---------------------------------------------------
\miniheader{Image Management}
%---------------------------------------------------

\begin{figure}[htp!]
    \dirtree{%
.1 \$SCRATCH/.uenv-images.
.2 index.db.
.2 images.
.3 01edd4\dots76a5ab.
.4 store.squashfs.
.4 env.json.
.4 config.json.
.4 recipe.
.5 \dots.
.4 extra.
.5 reframe.yaml.
.3 e7b0d9\dots47c377.
.4 \dots.
.3 e7e508\dots254742.
.4 \dots.
}


    \caption{The directory layout of a uenv \emph{repository}.}
    \label{fig:repo-path}
\end{figure}

\begin{figure*}[htp!]
    \begin{center}
        \begin{tikzpicture}[
    node distance=5cm and 3cm,
    %every node/.style={font=\sffamily},
    box/.style={draw, thick, minimum width=3.5cm, minimum height=2.5cm, align=center},
    arrow/.style={-{Latex[length=3mm]}, thick}
  ]

  \node[database,label=below:registry,database radius=1cm,database segment height=0.5cm] (rego) {};
  \node[database,label=below:repository,database radius=1cm,database segment height=0.5cm] at (6,-1) (repo) {};
  \node[alice,mirrored,label=below:user, minimum size=1.5cm] at (12,0) (user) {};

  \draw[arrow] (1.2,-0.5)  -- node[above] {image pull} (4.8,-0.5);
  \draw[arrow] (1.2,0.6)   -- node[above] {image find} (11.0,0.6);
  \draw[arrow] (7.5,-0.5)  -- node[above] {image ls} (11.0,-0.5);
  \draw[arrow] (7.5,-1.2)  -- node[above] {image inspect} (11.0,-1.2);
  \draw[arrow] (-1,-1) to[out=120, in=180,looseness=2] (-2,-2)  to[out=0,in=-100,looseness=2] (rego.south west);
  \draw[arrow] (5,-2) to[out=120, in=180,looseness=2] (4,-3) to[out=0,in=-100,looseness=2] (repo.south west);

  \node at (-2,-2.35) {image copy};
  \node at (-2,-2.75) {image delete};

  \node at (4,-3.35) {image add};
  \node at (4,-3.75) {image rm};
\end{tikzpicture}

    \end{center}
    \caption{The uenv image commands are used to manage uenv images in registries and repositories, and to query information about uenv images.}
    \label{fig:uenv-manage}
\end{figure*}

In~\sect{sec:cicd} the CI/CD pipeline that deploys uenv images to a container registry was described.
Users need to download images to local storage before they can use them on a cluster.
The uenv CLI provides a suite of functionality through the \lst{uenv image} command for querying, downloading, and other image management tasks, illustrated in~\fig{fig:uenv-manage}.

There are two locations where uenv images are stored:
\begin{itemize}
    \item \emph{remote}: a \emph{registry} is a container registry from which users can pull images to local storage;
    \item \emph{local}: a \emph{repository} is a directory on a local file system, from which uenv can be run.
\end{itemize}

Local storage for uenv images is a \emph{repository} -- a directory with an SQLite database at the root, and uenv images stored in an images sub-directory.
The database \lst{index.db} at the root of the repository allows for mapping the hash of each uenv to meta data like name, version, micro-architecture and target cluster.
Each image is stored in a path that matches its hash, with the following information:
\begin{itemize}
\item  \lstinline{store.squashfs}: the SquashFS file;
\item  \lstinline{env.json}: information about the mount point, uenv description, and view descriptions (their name, and the environment variable updates that they aplly).
\item  \lstinline{recipe}: the original recipe used to build the image;
\item  \lstinline{extra/reframe.yaml}: description of the testable features the uenv implements, and how to configure the environment to test them.
\end{itemize}

The uenv CLI tool uses the domain-specific language \lst{name/version:tag@system\%uarch} to describe uenv.

Users can query images in the registries and repositories using the \lst{uenv image find} and \lst{uenv images ls} commands respectively, for example the following will search for matching images in the users local repository:
\lstinputlisting[language=bash]{code/uenv-ls.sh}

The \lst{uenv image pull} command is used to pull an image from a remote registry.
The following example searches for all vasp images built for the system Daint, pulls one of them, then uses the \lst{image ls} command to show the downloaded image in the local repository:
\lstinputlisting[language=bash]{code/uenv-findpull.sh}

It is possible add and remove images from a local repository using the \lst{uenv image add} and \lst{uenv image rm} commands respectively.

Copying and deleting images in a registry using \lst{uenv image copy} and \lst{uenv image delete} respectively requires permission, which is restricted to the team at CSCS responsible for image deployment.
The \lst{image copy} command is used to deploy by copying images from the \emph{build} namespace (where the CI/CD pipeline pushes them after), to the \emph{deploy} namespace:
\lstinputlisting[language=bash]{code/uenv-deploy.sh}
Note that only one \squashfs image is stored after the two copies in the example above -- each copy creates a new label that is attached to the original \squashfs image.

%---------------------------------------------------
\miniheader{Loading uenv}
%---------------------------------------------------

Before describing how users can load uenv using the uenv CLI, we first define what it means to load a uenv environment.
Loading a uenv provides the software in the \squashfs by mounting the squashfs image and setting environment variables.

The uenv CLI uses the \lst{squashfs-mount} command line utility -- a small \lst{setuid} executable that creates a new mount namespace, mounts the SquashFS file through \lst{libmount}, drops privileges and executes a given command.
The following example starts a bash shell with the Squashfs files \lst{img1.sqfs} and \lst{img2.sqfs} are mounted at \lst{/mnt1} and \lst{/mnt2} respectively:
\lstinputlisting[language=bash]{code/squashfs-mount.sh}
The utility is open source, available on GitHub\footnote{\href{https://github.com/eth-cscs/squashfs-mount}{\lst{github.com/eth-cscs/squashfs-mount}}}, and includes RPMs for installation on Cray EX.

The second part of loading an environment is to set any relevant environment variables.
Each uenv image can provide any number of \emph{views}, which are defined defined in the \lst{meta/env.json} file in the \squashfs file.
Each view is a list of environment varaiables and modifications to make to them, which are applied to the environment variables set in the calling environment.

There are two categories of environment variables:
\begin{itemize}
    \item \emph{Scalar} are environment variables with a single value, for example \lst{CUDA_HOME}, \lst{BOOST_ROOT}, and \lst{TERM}.
        Each scalar variable in a view is a key-value pair, where the key is the variable name, and value is a string or \lst{null}.
        If the value is a string the variable is set (or overriden) to the new value, and if it is \lst{null} the variable is unset.
    \item \emph{Prefix} are environment variables that represent a colon-separted list of paths where the order of paths is significant, for example \lst{PATH}, \lst{LD_LIBRARY_PATH}, and \lst{PKG_CONFIG_PATH}.
        Prefix variables are representated as an ordered set of \emph{updates}, where each update is one of \emph{set}, \emph{prepend}, \emph{append}, \emph{unset}.
\end{itemize}

There are are two views that are automatically generated by Stackinator when the view is built:
\begin{itemize}
    \item\lst{modules}: prepends the module path inside the the uenv to the \lst{MODULE_PATH} environment variable, effectively making modules for packages in the uenv visible.
    \item\lst{spack}: sets environment variables that: point to the Spack configuration for all packages installed inside the uenv; provide the url for the Spack repository; and the branch/commit of Spack that was used.
        Together these variables provide all information required by users to use Spack to build their own software using the packages inside the uenv.
\end{itemize}

Recipe authors can define additional views, that configure the environment similarly to loading a set of modules.
For example, the GROMACS uenv images provide three additional views:
\begin{itemize}
    \item \lst{gromacs}: set all environment variables so that the GROMACS executable is in the path and ready to use.
    \item \lst{plumed}: set all environment variables so that a version of GROMACS with the popular PLUMED plugin is in the path and ready to use.
    \item \lst{develop}: set all environment variables so that all of the libraries, compilers and tools used to build GROMACS are ready to use, without GROMACS, for users who want to build a custom version of GROMACS.
\end{itemize}

%---------------------------------------------------
\miniheader{Using uenv}
%---------------------------------------------------

The uenv CLI provides two methods for using uenv images:
\begin{itemize}
\item\lst{uenv run}: runs a command in the uenv environment before returning to the calling shell.
\item\lst{uenv start}: starts a shell with the uenv environment loaded, for interactive use.
\end{itemize}

The \lst{uenv run} command is the most flexible, because it can be used interactively and in scripts including sbatch scripts.
The examples below show how it can be used to ``wrap'' individual calls in an environment:
\lstinputlisting[language=bash]{code/uenv-run.sh}
In practice there are no performance overheads from wrapping each launch using \lst{uenv run} -- indeed it is much faster than loading and swapping modules then executing software that is installed on a parallel filesystem.
Even when a \squashfs image is stored on Lustre, it is faster and more reliable than accessing the equivalent software stack installed directly on Lustre, because only one file is accessed, and \squashfs efficiently caches files and meta data in memory.

The \lst{uenv start} command is for interactive use cases, for example building software; testing and development; using debuggers and profilers; and using Python or Julia REPLs.
In the next example, the \lst{prgenv-gnu} uenv image is used to create and test a PyTorch virtual environment on a Grace-Hopper node.
The version of Python in the uenv is available through the \lst{default} view:
\lstinputlisting[language=bash]{code/uenv-start.sh}

%---------------------------------------------------
\miniheader{Lessons Learnt}
%---------------------------------------------------

The first version of uenv was written in Python, and was able to modify the calling environment similarly to modules.
For example, the following command would set environment variables for a uenv that was running:
\lstinputlisting[language=bash]{code/uenv-old-view.sh}

A full rewrite of uenv was conducted, based on the lessons learnt from this first version.

The first lesson, which is somewhat subjective, is that Python was not an ideal language for deploying to production for the following reasons:
\begin{itemize}
    \item the call to Python was designed to use Python 3.6 installed as part of SUSE, and had to be isolated;
    \item the project was deployed as a set of directories containing the Python implementating
    \item language features like the lack of type safety and exception-based error handling made implementing a robust, error-free code difficult as the size of the implementation increased.
\end{itemize}

This was simplified greatly by installing a single statically-linked C++ executable, written in C++20 to take advantage of modern language features for error handling, filesystem operations, and sanitizers.

The second lesson was that being able to modify the calling environment over-complicated the implementation, and enabled some bad practices.

In order to modify the calling environment, the \lst{uenv} command was bash function that forwarded the arguments to the Python implementation: \lstinline{echo "$($UENV_CMD $flags "$@")"}, where \lstinline{UENV_CMD} is the path of the implementation.
The implementation would process the arguments, and print a series of shell commands to stdout, which the calling bash function would then exec.
For example, the command \lstinline{uenv view default} might echo the following to stdout:
\lstinputlisting[language=bash]{code/uenv-old-echo.sh}

The wrapper-based implementation made it difficult to debug, required further wrappers inside calls to \lstinline{uenv start} and \lstinline{uenv run}, and made support for different shells difficult -- the original implementation only supported bash.
The new implementation only allows setting views when the uenv is loaded, which is performed by creating a new \lstinline{const char* environ} array that defines the new environment, and forwarding this to \lstinline{execvpe}.
This has the benefits:
\begin{itemize}
    \item the implementation is significantly simplified;
    \item it is shell agnostic - there is no bash-specific code anywhere in the implementation.
\end{itemize}

The final benefit is more subtle: specifying the target environment up front is declarative.
User tickets are easier to debug because the views they are loading are explicitly listed as flags to the uenv call, and all environment modifications are captured in the logs of the \lstinline{uenv start}, \lstinline{uenv run} or SLURM \lstinline{srun --uenv=? --view=?} calls.
