User-facing software -- libraries, tools, applications and programming environments tuned for the node and network architecture -- is a key part of HPC centers' service offering.
Teams that maintain and support this software face challenges: providing a stable software platform for users with long running projects while also providing the latest versions of software for developers; giving full responsibility to build, modify and deploy the whole software stack to staff who do not have root access; and reproducible deployment based on GitOps practices.

CSCS addresses these challenges on Alps by using small independent software environments called uenv, which deploy from text-file recipes without requiring installation of the Cray Programming Environment.
This paper discusses installing communication libraries from HPE and NVIDIA with \slingshot support; the CI/CD pipeline that builds uenv and deploys them in a container registry; and the command line tools and SLURM plugin that interface users with the software environments.
We demonstrate diverse use cases such as JupyterHub, summarize the user and support team experience, and document how to build and deploy CPE containers.

