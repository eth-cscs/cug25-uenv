A differentiating feature of Alps compared to other HPE Cray-EX deployments is how multi-tenancy is implemented and supported.
Instead of deploying a single large SLURM cluster, the system is partitioned into non-overlapping SLURM and Kubernetes clusters, each customised for specific use cases and community requirements.
The combination of tenants and use cases requires distributing responsibility for providing and supporting software environments to many individuals and teams inside CSCS.

Ideally software support staff would have end to end responsibility: the ability to install all software as users without interrupting the operation of the system, using automated CI/CD pipelines.
Furthermore, to continue scaling support to more use cases and systems, the tooling and methods used need to:
\begin{itemize}
    \item reduce the dependence of software stacks on the base OS image installed on each tenant
    \item decouple (and simplify) software environments
    \item automate deployment
    \item and, support users requiring custom software stacks.
\end{itemize}

CSCS has found that there is no silver bullet that meets all of these requirements.
For example, HPC containers allow users to bring software stacks built elsewhere, they offer excellent isolation between environments, and container runtimes can inject optimized communication libraries making containerised software resilient to changes to the underlying system.
But supporting containers alone is not sufficient -- we also need to provide software optimised for the target platform, and support users building their own optimized containers.
In other words, while providing a robust container runtime with SLURM integration reduces dependency on the base OS and decouples software stacks, .

At CUG 2023 CSCS presented \stackinator~\cite{uenv2023}, a tool for building isolated and reproducible software environments as \squashfs images.
The software stacks have no dependency on the Cray Programming Environment (CPE), and use custom \spack packages for installing the \craympich implementation of MPI and its dependencies as stand alone packages.
Over the last two years CSCS have continued to develop \stackinator, and the user-facing tools for interacting with the software stacks, which we have dubbed \emph{uenv}.
This paper focusses on the full end to end process and tooling that CSCS has developed to deploy uenv, how users interact with them, and their impact on support staff.

In \sect{sec:objectives} the requirements discussed above will be enumerated as objectives with measurable aims.
Then the tools and processes used to achieve our objectives will be introduced in three stages: building environments; deploying environments; and the user experience for users of the deployed environments in \sect{sec:methods}.
How CSCS continues to provide CPE as a container, with recipes and lessons learnt for building CPE in a container will be covered in \sect{sec:cpe}.
Finally we will describe some interesting use cases for uenv in \sect{sec:usecases}, and discuss the maintenance overheads and issues with our approach in~\sect{sec:discussion}.

We note that while much of the implementation is CSCS-specific, all of the software and pipelines discussed in this paper are in permissibly licensed open source repositories that can be used as is, or used for tips and guidance in developing solutions at other sites.
