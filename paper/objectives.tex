We first establish concrete objectives that define the requirements for deploying software on HPC systems.
These objectives can be used to evaluate the effectiveness of our proposed solutions.
The objectives span technical, operational, and organizational aspects of software deployment.

%------------------------------------------------------------------------------
\subsection{Provide optimized software}
%------------------------------------------------------------------------------

Libraries that implement inter-node communication, for example MPI and NCCL, need to be optimized for the Slingshot 11 network in HPE Cray-EX systems.
The method used by \stackinator to install \craympich using \spack~\cite{gamblin:sc15} outside the CPE was demonstrated in~\cite{uenv2023} 

To support more applications, including ML/AI and commercial applications , similarly robust methods are required for installing packages like OpenMPI, NCCL, \nvshmem, \cufftmp and \cusolvermp and with libfabric support:
\begin{itemize}
    \item \textbf{aim}: we can deploy OpenMPI with Slingshot 11 support;
    \item \textbf{aim}: build libfabric and CXI from source.
    \item \textbf{aim}: we can deploy key NVIDIA communication libraries on Slingshot 11;
\end{itemize}

%------------------------------------------------------------------------------
\subsection{Gitops deployment}
%------------------------------------------------------------------------------

With GitOps, infrastructure and environments are defined as code in version control systems like Git.
This enables automated, reproducible deployments through CI/CD pipelines, while providing transparency, rollback capabilities, and a single source of truth for the entire system configuration.

We define the following criteria for a GitOps workflow for deploying software on a HPC system:
\begin{itemize}
    \item \textbf{aim}: all software stacks are defined as text file recipes in a git repository;
    \item \textbf{aim}: a CI/CD pipeline builds, tests and deploys uenv images using triggers on the git repository;
    \item \textbf{aim}: each pipeline build deploys unique artifacts;
    \item \textbf{aim}: artifacts can be rebuilt from recipe.
\end{itemize}

%------------------------------------------------------------------------------
\subsection{Decouple environments}
%------------------------------------------------------------------------------

Many use cases on Alps require stable environments that do not change over the duration of multi-year projects,
while others require updated versions of the same environments, or frequent rolling release style deployment to keep pace with rapidly changing software stacks (for example ML/AI and development projects).
%This requires decoupling of software environments, so that new software can be deployed alongside existing software with no side effects, which is summarised with the following aims:
To support all use cases on the same system, we set the following requirements:
\begin{itemize}
    \item \textbf{aim}: existing workflows based are unaffected by releasing new versions of the uenv;
    \item \textbf{aim}: minimise the chance that uenv are ``broken'' by updates to the underlying system, while being able to redeploy them quickly if there is a breaking change;
    \item \textbf{aim}: uenv that provide the latest versions of software (e.g. with pre-releases of cray-mpich or the latest PyTorch version) can be deployed immediately after their release.
\end{itemize}

%------------------------------------------------------------------------------
\subsection{End to end responsibility for software support teams.}
\label{sec:objective-e2e}
%------------------------------------------------------------------------------

The team responsible for supporting software environments should have full responsibility for its deployment.
Building and deploying software should not require root privileges, modification to the system itself or the involvement of system administrators:
\begin{itemize}
    \item \textbf{aim}: staff who support applications and programming environments should have end to end responsibility;
    \item \textbf{aim}: user communities can manage their own software deployments using the same tools;
    \item \textbf{aim}: individual users can build and share their own uenv images.
\end{itemize}

